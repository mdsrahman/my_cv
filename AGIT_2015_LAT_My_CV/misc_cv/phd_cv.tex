\documentclass[margin,line,pifont,palatino,courier]{res}

\usepackage{pifont}
\usepackage[latin1] { inputenc}
\usepackage{hyperref,verbatim}
\usepackage{amsfonts,amsmath}
%\topmargin .5in 
%\oddsidemargin -.5in
%\evensidemargin -.5in
%\textwidth=6.0in
 \textheight=9.0in  
%\itemsep=0in
%\parsep=0in
\usepackage{fancyhdr}
%\topmargin=0in
%\textheight=8.5in
\pagestyle{fancy}
\renewcommand{\headrulewidth}{0pt}
\fancyhf{}
%\cfoot{\thepage}
%\lfoot{\textit{\footnotesize Research Statement}}
\rfoot{{\footnotesize Curriculum Vitae, Md. Shaifur Rahman, \thepage}}

  
\newenvironment{list1}{
  \begin{list}{\ding{113}}{%
      \setlength{\itemsep}{0in}
      \setlength{\parsep}{0in} \setlength{\parskip}{0in}
      \setlength{\topsep}{0in} \setlength{\partopsep}{0in}
      \setlength{\leftmargin}{0.17in}}}{\end{list}}
\newenvironment{list2}{
  \begin{list}{$\bullet$}{%
      \setlength{\itemsep}{0in}
      \setlength{\parsep}{0in} \setlength{\parskip}{0in}
      \setlength{\topsep}{0in} \setlength{\partopsep}{0in}
      \setlength{\leftmargin}{0.2in}}}{\end{list}}

\begin{document}

\name{Md. Shaifur Rahman \vspace*{.1in}}

\begin{resume}

\section{\sc Contact Information}

\vspace{.05in} 
\begin{tabular}{@{}p{3.1in}p{2.3in}}
PhD Student, Department of Computer Science\\
Stony Brook University, NY 11794-4400. \\
Cellphone: 631-949-6815 \\
Email:
\href{mailto:mdsrahman@cs.stonybrook.edu}{mdsrahman@cs.stonybrook.edu}\\
URL: 
\href{http://shaifur.com}{http://shaifur.com}\\
\end{tabular}
 
\section{\sc Research Interests}
Wireless Sensor Network, Systems \& Networking, Artificial Intelligence,
Computational Biology

\section{\sc Education}
\textbf{PhD in Computer Science (2013 to present)} \\
\href{http://www.cs.sunysb.edu/}{Stony Brook University, NY}.  \\
CGPA: 3.54/4.00  
\\ \\
\textbf{M.Sc. in Computer Science \& Engineering, 2013} \\
\href{http://www.buet.ac.bd/cse/}{Bangladesh University of Engineering \&
Technology}
\\
CGPA: 3.75/4.00  
\\ \\
\textbf{B.Sc. in Computer Science \& Engineering, 2009} \\
\href{http://www.buet.ac.bd/cse/}{Bangladesh University of Engineering \&
Technology} \\
CGPA: 3.92/4.00\\
Position: Ranked 4th in a class of 127 students

\section{\sc Honors and Awards}
\textbullet{} Special fellowship of CS department, Stony Brook
University, 2013.
\\
\textbullet{ } Dean's List Award for academic excellence in all levels of
B.Sc.
\\
\textbullet{ } University Merit Scholarship for academic excellence in all
levels of B.Sc. \\

\section{\sc Research Experience}
1. \textbf{Path Planning Algorithm for Mobile Data Collector in Wireless
Sensor Network}\\ {[}2011 to 2013{]} 
Worked with
\href{http://www.buet.ac.bd/cse/faculty/facdetail.php?id=mahmudanaznin}{Dr.
Mahmuda Naznin} and
\href{http://www.buet.ac.bd/cse/faculty/facdetail.php?id=yusufsarwar}{Dr. Yusuf Sarwar Uddin} to minimize the tour-length of a mobile data collector that ferries data from sensor nodes to a sink in a network. Our algorithm takes as input a $TSP$-tour, generates a \textit{Label-covering} tour and contracts the path by linear-shortcutting
method up to a point where one or more points in the tour that are called critical point, can
never be skipped. The complexity of the algorithm except the computation of the $TSP$ tour is
$O(n^3)$ where $n$ is the number of nodes covered by the data collector. The resulting tour
reduces path length, improves
data delivery latency and increases network lifetime. An extensive simulation in Castalia framework
of OMNET++ simulator validates our claim. We also tweaked the MAC layer protocol for communication
between the mobile element and static sensor node to save energy.\\ \\
2. \textbf{Application of Ant Colony Optimization in Energy-efficient Dynamic Source Routing in 
WSN}\\ 
{[}2008 -- 2009{]} Worked with Dr. Mahmuda Naznin to test performance of different $ACO$ algorithms
to create on-demand routing paths in WSN for energy-efficient source routing. Simulation result in
$NS$-$2$ showed that overloading the computation with a lot of system parameters does not render
much gain in network lifetime.
Instead, the naive $ACO$ algorithm with few simple parameters to reduce the number 
of cross-road nodes in the routing paths outperforms all other versions in increasing network
lifetime.
\section{\sc Publications}
1. Md. Shaifur Rahman and Mahmuda Naznin, ``Shortening the Tour-length of a Mobile Data Collector
  in the WSN by the Method of Linear Shortcut''. In Proceedings of the
  \textit{$15^{th}$ Asia-Pacific Web Conference (APWEB'13)}, 2013, Sydney,
  Australia (\textit{LNCS, Springer})
\section{\sc Graduate Level Courseworks in Stony Brook University}
\begin{tabular}{@{}p{2.3in}p{3in}}
\begin{list1}
\item Artificial Intelligence
\item Analysis of Algorithms 
\item Computational Biology
\end{list1}
\end{tabular}

\section{\sc Graduate Level Courseworks in M.Sc.}
\begin{tabular}{@{}p{2.3in}p{3in}}
\begin{list1}
\item VLSI Layout Algorithms
\item Advanced Database Systems 
\item Bioinformatics Algorithms 
\end{list1}
&
\begin{list1}
\item Wireless Resource Management 
\item Wireless Ad Hoc Networks 
\item Neural Networks 
\end{list1}
\end{tabular}

%\section{\sc Publication}
%{[}Under Review{]}\\
%\textbullet{ }Md. Shaifur Rahman and Mahmuda Naznin, \textit{Path Contraction of Mobile Data
%Collectors by the Method of Linear Shortcutting}, ACM Transaction on Sensor Networks.

\section{\sc Professional Experience}
\begin{tabular}{@{}{l}{r}}
\textbf{Department of Computer Science \& Engineering, }\\
\textbf{Bangladesh University of Engineering \& Technology} \\
Assistant Professor (On Leave, \textit{June, 2013
to Present})\\
Lecturer (\textit{June 2009 to May 2013})
\end{tabular}
\section{\sc Teaching Experience}
\textbf{ Teaching Assistant in Stony Brook University: } CSE308: Software
Development (Fall-2013) and CSE215: Foundations of Computer Science
(Spring-2014)\\
 \textbf{Theory Courses Instructed:
}VLSI Design, Artificial Intelligence, Theory of Computation\\
\textbf{Lab Courses Instructed: }Technical Writing \& Presentation, Artificial Intelligence,
Operating Systems, VLSI Design, Database, Microprocessor, Digital Logic Design, Pattern Recognition etc.\\
\section{\sc Training \& Workshop}
\textbullet{ }Cisco CCNA Instructor's Program for Module 1, 2, 3 \& 4 conducted by the Cisco
Networking Academy in BUET.\\
\textbullet{ }Teacher's Appreciation Program conducted by Directorate of Advisory, Extension \&
Research Services, BUET.

\section{\sc Organizing Experience}
\textbullet{ }Member of Organizing Committee, Workshop on Algorithms \& Computation-
\href{http://www.buet.ac.bd/cse/walcom2010/}{WALCOM-2010}
\&
\href{http://www.buet.ac.bd/cse/walcom2012}{WALCOM-2012}\\
\textbullet{ }Trainer \& Organizer of Automated SQL Learning \& Evaluation workshop 2012, sponsored
by Ministry of Education, Bangladesh\\
\textbullet{ }Trainer of short-courses in \href{http://www.buet.ac.bd/cse/iac}{Bangladesh-Korea
Information Access Center}, BUET\\
\textbullet{ }Trainer of Advanced Networking Training Program(Cisco ICND-1 \& ICND-2) for employees
of IT department, Bangladesh Central Bank

\section{\sc Projects Completed in Graduate \\ Classes}
\textbf{Pre-Overlapper}\\
We implemented a pre-processing step for the
\href{http://schatzlab.cshl.edu/teaching/2012/pacbioasm.shtml}{De Novo
genome assembly of PacBio/ Nanopore short-reads of Bacteria} as part of
the project work done in CSE549 (Computational Biology) course.

\section{\sc Projects Completed in Undergraduate Classes}
\textbf{3D Golf Game in OpenGL \& C++}\\
As part of the project assigned in the Graphics Lab, we implemented 3D Golf Game with picturesque
terrain of grass, pond, mud etc. and projectile physics and collision detection for game score.\\
\textbf{Cellphone-based Voice-controlled Operation of Home Appliances}\\
We captured the voice
from cellphone and analyzed it using Microsoft's relevant MSDN library. Using ATmega32 
micro-controller, different home appliances
like light-bulb, fan, heater etc was turned on/off and their intensity of operation was
controlled.\\
\begin{comment}
\textbf{Inventory Management System of Onik Ltd.}\\
We studied the system of cellphone selling company, developed extensive UML and prototype based on
it, validated the design and completed the development of the system using ASP,NET, Oracle and
Crystal Report.\\
\end{comment}
\textbf{4-bit Microprocessor}\\
A simple microprocessor that was simulated in Circuit-maker and later implemented in hardware. The
feature included execution of $28$ instructions of $80 \times 86$ processor family, memory
protection, multiprogramming etc.\\
\textbf{NACHOS Virtual OS Implementation}\\
As part of the task in the operating system lab, we implemented Multiprogramming,
Process Management, Console, and elementary system calls of virtual operating system NACHOS.\\
\textbf{C Compiler}\\
We implemented a complete compiler for $C$ program using $Lex$ and $Yacc$ as part of the task in
compiler lab.\\
%\newpage
\section{\sc Skills}
\textbf{Programming Language}\\
C/C++, Java, Prolog, Python\\
\textbf{Web Development}\\
DHTML, PHP, JSP\\
\textbf{Database}\\ 
Oracle, MySQL\\
\textbf{Other Tools}\\
OpenGL, PSPICE, Microwind, Verilog HDL\\
\textbf{Technical Writing \& Simulation Tools}\\
\LaTeX{}, GNUPlot, MatLab, Network Simulator 2 \& 3, OMNET++, OPNET
\section{\sc Co-curricular Activities}
\textbf{Debating: }Participated in Model United Nations Debate - 2002, National Debate Championship
- 2000, 2001 \& 2002\\
\textbf{AIDS Awareness Campaign: }Participated in Countrywide AIDS Awareness Campaign for Youths
2002--2004, sponsored by UNICEF.
\section{\sc References}
\textbf{Dr. Mahmuda Naznin}\\
Associate Professor\\
Department of Computer Science \& Engineering \\
Bangladesh University of Engineering \& Technology. \\
Email: \href{mailto:mahmudanaznin@cse.buet.ac.bd}{mahmudanaznin@cse.buet.ac.bd}\\ 
Web-page: \href{http://teacher.buet.ac.bd/mahmudanaznin}{http://teacher.buet.ac.bd/mahmudanaznin}
\\
\\
\textbf{Dr. Saidur Rahman}\\
Professor\\
Department of Computer Science \& Engineering \\
Bangladesh University of Engineering \& Technology. \\
Email: \href{mailto:saidurrahman@cse.buet.ac.bd}{saidurrahman@cse.buet.ac.bd}\\ 
Web-page: \href{http://teacher.buet.ac.bd/saidurrahman}{http://teacher.buet.ac.bd/saidurrahman}
\\
\\
\textbf{Dr. I.V. Ramakrishnan}\\
Professor \& Graduate Program Director\\
Department of Computer Science\\
Stony Brook University, NY-11794-4400 \\
Email: \href{mailto:ram@cs.sunysb.edu}{ram@cs.sunysb.edu}\\ 
Web-page:
\href{http://www.cs.sunysb.edu/~ram/}{http://www.cs.sunysb.edu/$\sim$ram/}
\end{resume}
\end{document}
