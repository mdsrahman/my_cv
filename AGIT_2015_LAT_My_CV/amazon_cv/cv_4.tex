%%%%%%%%%%%%%%%%%%%%%%%%%%%%%%%%%%%%%%%%%
% Medium Length Professional CV
% LaTeX Template
% Version 2.0 (8/5/13)
%
% This template has been downloaded from:
% http://www.LaTeXTemplates.com
%
% Original author:
% Trey Hunner (http://www.treyhunner.com/)
%
% Important note:
% This template requires the resume.cls file to be in the same directory as the
% .tex file. The resume.cls file provides the resume style used for structuring the
% document.
%
%%%%%%%%%%%%%%%%%%%%%%%%%%%%%%%%%%%%%%%%%

%----------------------------------------------------------------------------------------
%	PACKAGES AND OTHER DOCUMENT CONFIGURATIONS
%----------------------------------------------------------------------------------------

\documentclass{resume} % Use the custom resume.cls style
 
\usepackage[left=0.75in,top=0.2in,right=0.75in,bottom=0.2in]{geometry} %
% Document margins
\usepackage{marvosym,ifsym,hyperref,amssymb,verbatim,titlesec,fontawesome} 

\name{Md Shaifur Rahman} % Your name
\address{\Keyboard{ }
\href{http://shaifur.com}{http://shaifur.com}\quad 
\faLinkedinSign :
\href{http://www.linkedin.com/in/shaifur}{http://www.linkedin.com/in/shaifur}}
% address
\address{\textifsymbol{18}{ }1147 N. Country Rd., Stony Brook, NY-11790} % Your
\address{\Mobilefone{ } \textbf{+1-631-949-6815}\quad \Letter{ }
mdshaifur.rahman@stonybrook.edu}
% Your phone number and email

\begin{document}
 \titlespacing{\section}{0pt}{\parskip}{-\parskip}
%----------------------------------------------------------------------------------------
%	EDUCATION SECTION
%----------------------------------------------------------------------------------------

\begin{rSection}{Education}
{\bf PhD in Computer Science} \hfill {August, 2013 -- present} \\ 
Stony Brook University, NY \\ 
CGPA: 3.65/4.00 \\[4pt]
{\bf M.Sc. in Computer Science} \hfill {2009 -- 2012} \\ 
Bangladesh University of Engineering \& Technology, Bangladesh \\
CGPA: 3.75/4.00 \\[4pt]
{\bf B.Sc. in Computer Science} \hfill {2004 -- 2009} \\ 
Bangladesh University of Engineering \& Technology, Bangladesh \\ 
CGPA: 3.92/4.00 (Class position: 5th)
\end{rSection}
%----------------------------------------------------------------------------------------
%	TECHNICAL STRENGTHS SECTION
%----------------------------------------------------------------------------------------
\begin{rSection}{Key Skills}
\textbf{Programming Languages: }C, C++, Java, Python, Prolog, Assembly x86 \\ 
\textbf{Database-related Skills: } Oracle, MySQL, ERD, XML, XPath\\ 
\textbf{Operating Systems: }Linux/Unix, Shell scripting, Windows\\ 
\textbf{Other Tools \& Languages: }Eclipse, Github, \LaTeX{}, Matlab, HTML,
PHP, NS-3, Wireshark, Verilog\\
\textbf{Specialized Training: }Cisco CCNA Module 1--4 Academic Trainer 
\end{rSection}  
%----------------------------------------------------------------------------------------
%	WORK EXPERIENCE SECTION
%----------------------------------------------------------------------------------------
\begin{rSection}{Academic Projects}
\begin{rSubsection}{Stony Brook University}{}{}{}
\item \textit{Implementation of Chain Replication} (van Renesse et al.) as a
high throughput and highly available fail-stop server system using Python and Java
languages as part of \textit{Asynchrounous Systems} course in Fall-2014 semester
\item \textit{Spatial Analysis of WiFi Data} using Wireshark and WLAN monitor
mode in Linux as part of \textit{Wireless \& Mobile Network} course in Fall-2014
semester 
\item \textit{Effect of Cache-size \& Pending Interest Table
aggregation in Named Data Content} using emulated test-bed on 
\href{http://www.emulab.net/}{http://www.emulab.net/} as part of
\textit{Fundamental of Computer Networks} course in Spring-2014 semester
\item \textit{Efficient Pre-overlapper: a preprocessing step for the De Novo
genome assembly of PacBio/Nanopore shortreads of bacteria genome} using C/C++ as
part of \textit{Computational Biology} course in Fall-2013 semester
\item \textit{Implementation of Learning Algorithms (Decision Tree, Q-learning
etc.) in Pacman Maze Game} using Python as part of \textit{Artificial
Intelligence} course in Fall-2013 semester
\end{rSubsection}  

\begin{rSubsection}{Bangladesh University of Engineering \& Technology}{}{}{}
\item \textit{Implementation of multiprocessing \& rudimentary filesystem in
NACHOS Virtual OS} using C/C++ and MIPS as part of \textit{Operating Systems}
course
%\begin{comment}
\item \textit{ANSI C Compiler} using \textit{Lex} and
\textit{Yacc} as  specification tools as part of \textit{Compiler} course
\begin{comment}
\item \textit{A simple 4-bit Microprocessor based on 8086 architecture}
implemented in hardware using basic chips on breadboard in
\textit{Digital Systems Design} course and simulated in Verilog in \textit{VLSI
Design} course
\item \textit{Cellphone-based Voice-controlled Operation of Home Appliances}
using ATmega32 micro-controller and Microsoft MSDN speech-recognition API as
part of \textit{Computer Interfacing} course
\end{comment}
\item \textit{Management Tools for a software development firm} with provision
for Gantt-chart, code-repo etc. using Oracle 10g, Java \& JDBC as part of
\textit{Database} course
\end{rSubsection}
\end{rSection}
%----------------------------------------------------------------------------------------
%	WORK EXPERIENCE SECTION
%----------------------------------------------------------------------------------------
\begin{rSection}{Experience}
\begin{rSubsection}{Stony Brook University}{August, 2013 -- 
Present}{Teaching Assistant \& Research Assistant}{}
\item Researching high capacity wireless backbone in cellular networks and data
centers via Free Space Optics, algorithmic issues related to deployment of base
stations, traffic engineering in the heterogenous networks etc.
\item Took recitation classes, supervised student projects and graded exam
scripts in undergrad courses like Software Development, Theory of Database
Systems etc.
\end{rSubsection} 
%------------------------------------------------ 
\begin{rSubsection}{Bangladesh University of Engineering \&
Technology}{May, 2009 -- July, 2013}{Lecturer(2009 -- 2013) \& Assistant
Professor (2013)}{}
\item Instructed Undergrad classes like Object Oriented Programming, Database,
 VLSI Design, Computer Networks 
\end{rSubsection}
%------------------------------------------------
\end{rSection}
%----------------------------------------------------------------------------------------
%	Publications
%----------------------------------------------------------------------------------------
\begin{rSection}{Publication}
 Md. Shaifur Rahman and Mahmuda Naznin, ``Shortening the Tour-length of a Mobile Data Collector
  in the WSN by the Method of Linear Shortcut''. In Proceedings of the
  \textit{$15^{th}$ Asia-Pacific Web Conference (APWEB'13)}, 2013, Sydney,
  Australia (\textit{LNCS, Springer})
\end{rSection}
%----------------------------------------------------------------------------------------
\end{document}
